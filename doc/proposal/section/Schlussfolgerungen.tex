Das Ziel des Projektes ist es, die für Capturing relevante Software des
Betriebssystem FreeBSD zu untersuchen, die Probleme, die zur Reduzierung der
Capturing-Performance führen, herauszufinden und aufgrund der gefundenen
Problemen den neuen (ringmap)-FreeBSD-Packet-Capturing-Stack zu erarbeiten,
zu implementieren und auszuwerten. Die neue Lösung basiert auf der für das 
Projekt vorausgesetzte Hardware (siehe Abschnitt \ref{sec:hwsw_voraus}).
%
\ifthenelse{\boolean{BRIEF}}{}{   
Die Diplomarbeit ist in drei Phasen abgelaufen: Einarbeitung, Implementierung
und Testen.  Während der \textbf{Einarbeitungszeit} habe ich viele neue
Kenntnise in Themen, die über das normales Informatiksstudium hinausgehen,
erworben.  Es handelt sich vor allem um Wissen über die Funktionsweise moderner
Netzwerkadapter und Betriebssystemkern- und -Treiber-Programmierung. Unter
Anderem habe ich dabei neue Konzepte über die Hardware-Gegebenheiten und
Software-Implementierung der modernen Interrupt- und DMA-Mechanismen (siehe
Kapitel Grundlagen, Abschnitt \ref{sec:adapter}) gelernt. \\\\
%
Außerdem wurden von mir mehrere Arbeiten~\cite{fabian_da, pcin10gb_paper,
perfev_paper, bpf_paper} über Capturing-Problematik durchgearbeitet mit dem
Ziel, die beim Capturing vorhandene Probleme zu analysieren, um die
Anforderungen für den neuen \emph{ringmap}-Packet-Capturing-Stack zu erstellen.
Für die Capturing-Performance-Probleme wurde in den Arbeiten hauptsachlich die
Anzahl der Paket-Kopie-Operationnen und die hohe Interrupt-Rate bei der
Verkehr-Erfassung als Gründe genannt. Als eine mögliche Lösung zur des
Interrupt-Overhead wird die Verwendung des
\emph{Polling}-Mechanismus~\cite{bsd_polling} vorgeschlagen.  Wegen der
Instabilität der BSD-Polling-Implementierung für
SMP-Kernel~\cite{bsd_devpoll_page} habe ich für den Entwurf des neuen Stacks
allerdings das normale Interrupt-Driven-Modell eingesetzt. Neben der Anzahl der
Kopie-Operationen, wurden im \emph{generic}-Capturing-Stack einige andere
Implementierungsdetails entdeckt, welche die Capturing-Performance bei hohen
Paket-Raten beeinflüssen können. Vor allem die Speicher-Allozierungen welche
der \emph{generic}-Treiber für jedes neue Packet durchführt, beeinflüssen die
Performanz negativ. Mit dem Ziel die herausgefundene
Capturing-Performance-Probleme zu eliminieren wurden die Anforderungen und den
Entwurf der Implementierung des neuen \emph{ringmap}-Packet-Capturing-Stack
erstellt (Kapitel Grundlagen, Abschnitt \ref{sec:aufgstel}).\\\\
%
In der \textbf{Implementierung}-Phase wurde der neue
\emph{ringmap}-Packet-Capturing-Stack für das Betriebssystem FreeBSD
programmiert (siehe Kapitel Implementierung). Dafür wurde der Code für
Adapter-Treiber, Systemaufruf-Funktionen und Libpcap-Library implementiert.
Während der Implementierung  habe ich tiefgehende Kenntnisse und viel
praktische Erfahrung in  Kernel- bzw. Treiberprogramming für
UNIX-artige-Systeme erworben.  \\\\
%
In der \textbf{Test}-Phase wurde der neue \emph{ringmap}-Packet-Capturing-Stack
getestet und mit dem \emph{generic}-Stack in Bezug auf leistungsfähigkeit
vergliechen. Für das Testen des \emph{ringmap}-Stack wurden unterschiedliche
Verkehr-Raten generiert, unterschiedliche Treiber-Parameter eingesetzt und
verschiedene Hardware für den Tets verwendet (siehe Kapitel
Leistungsbewertung). Für die Tests wurden \emph{Shell}-Skripte implementiert
die sowohl alle Testablüfe gesteuert haben als auch die Testergebnisse
ausgewertet und in Form von Tabellen und \emph{gnuplot}-Grafiken gespeichert
haben. 
}
\subsection{Erreichte Ziele}

\subsubsection{Verbesserung der Capturing-Performance}
Beim Einsatz des im Rahmen des Projektes implementierten
\emph{ringmap}-Stacks wird die Capturing-Performance unter gleichen Bedingungen
höher als beim \emph{generic}-Stack. Die \textbf{Systemload} beim Capturing mit
dem \emph{ringmap}-Stack ist deutlich geringer als mit dem \emph{generic}-Stack.
Bei allen durchgeführten Experimenten war die Systemload beim Capturing mit dem
\emph{ringmap} kleiner als $12\%$. Die Systemload bei der Verwendung von
\emph{generic}-Stack variiert von $13\%$ bis $100\%$. Bezüglich der
Systemload zeigt der \emph{ringmap} einen deutlichen Gewinn.\\\\
%
Anders sieht es mit den \textbf{Paketverlusten} aus. Beide Stacks zeigen
identisch gute ($100\%$) Paketerfassungsrate für den Fall wenn die Paketgröße
über 200 Bytes liegt. Bei Paketen kleiner als 200 Bytes, und daraus
verursachten höheren Paketraten, $> 450000 pkts/sec$, verliert der
\emph{generic}-Stack (mit 10MB BPF-Puffer-Größe) bis zu $100\%$ aller der
generierten Pakete. Selbst bei maximaler Belastung verliert der \emph{ringmap}-Stack 
weniger als $0.02\%$ der Pakete.
%
\subsubsection{Transparenz}
Alle auf \emph{libpcap} basierte Anwendungen wie \emph{tcpdump, wireshark,
etc\ldots} benötigen keine Änderungen um mit dem \emph{ringmap}-Stack die Pakete
zu erfassen. 
\ifthenelse{\boolean{BRIEF}}{}{   
\subsubsection{Stabilität und Benutzbarkeit}\label{sec:stab_und_func}
Der \emph{ringmap}-Capturing-Stack hat in der Tests-Phase (insgesamt mehr als
100 000 Tests mit den unterschiedlichen Treiber-Parameter) weder
\emph{Segmentation faults} noch keine \emph{Kernel panics} verursacht. Das
bedeutet, dass die für Kernelspace und Userspace (Libpcap) implementierte
Software sehr stabil ist.\\\\
%
Der \emph{ringmap} ist sehr einfach einzusetzen. Im Verzeichnis 
\verb+scripts/+ befinden sich zwei Shell-Skripts:
\begin{itemize}
	\item \verb+ringmap_build_and_load+: Kompiliert und installiert \emph{ringmap}-Software. 
	\item \verb+generic_build_and_load+: Bringt das System in den generischen Zustand. 
\end{itemize}
Dadurch, dass der Userspace-Code in der Libpcap-Library eingesetzt wurde,
erlaubt es allen Anwendungen, die auf Libpcap basieren, zum Beispiel \emph{tcpdump},
ohne Änderungen  Netzwerk-Pakete mit dem \emph{ringmap}-Stack zu erfassen
(siehe Einschränkungen im nächsten Abschnitt \ref{sec:einschr}).\\\\
%
Zudem lässt sich der \emph{ringmap}-Capturing-Stack, im Fall dass es auf dem 
System mehrere unterschiedliche Intel Gigabit Adapters gibt, durch die 
Eingabe der \emph{PCI-Device-ID} nur für einen ausgewählten Adapter einsetzen, 
sodass der \emph{generic}-\textbf{em}-Treiber die restliche Adapters steuert. 
Die \emph{PCI-Device-ID} des Adapter wird als Makrodefinition \verb+DEV_ID+ in der 
Datei \verb+fiveg_da.h+ eingegeben. Dadurch ist es möglich \emph{ringmap}- und 
\emph{generic}-Software auf einem System gleichzeitig zu benutzen.

}
\subsection{Einschränkungen des ringmap-Packet-Capturing-Stack}\label{sec:einschr}
Die Benutzung der \emph{ringmap}-Software verursacht auf dem Capturing-System
(bzw. für die Ausgewählte Adapter) folgende Einschränkungen:
\begin{description}
	\item [Kein TCP/IP Protokollstack während Capturing:] Für das 
		Capturing mit dem \emph{ringmap} muss das Netzwerk-Interface in
		Monitoring-Mode gesetzt werden. Aus diesem Grund besteht keine 
		Möglichkeit, das gleiche Interface gleichzeitig für Capturing 
		und Kommunikation zu nutzen.
	\item [Libpcap Einschränkungen:] Für das Anpassen der Libpcap an den \emph{ringmap}-Stack wurden einige Code-Stücke im
		Libpcap-Quell-Code auskommentiert oder ersetzt. Dadurch ist die
		Funktionalität der neuen \emph{ringmap}-Libpcap ist nicht mit
		der \emph{generic}-Libpcap identisch. Aus der
		\emph{generic}-Libpcap sind die Funktionalität entfernt, welche den
		BPF im Kernelspace anfordern. Außerdem wurde der
		\verb+to_ms+-Parameter~\cite{man_pcap} für die
		\verb+pcap_open_live()+-Funktion deaktiviert, aber aus 
		Kompatibilitätsgründen geblieben. Beim
		Capturing mit dem \emph{ringmap}-Stack blockiert der Userspace-Prozess wenn
		es keine neue Pakete mehr gibt, bis  weitere Pakete ankommen.
\end{description}
Alle genannte Einschränkungen sind  in dem Sinne nicht kritisch, dass 
sie sich eliminieren oder, für bestimmte Anforderungen, anpassen lassen. 
Der Source-Code von \emph{ringmap} ist auf Google-Code veröffentlicht 
und unter dem Namen \emph{ringmap} auf dem Server zu finden. An der  zukünftigen 
Weiterentwicklung des Projektes habe ich großes Interesse und biete meine Hilfe an.

\subsection{Zukünftige Themen}
\subsubsection{Performance-Vergleich mit dem Zero-Copy-BPF-Buffers}
Im Lauf der Test-Phase der Diplomarbeit, war der \emph{Zero-Copy-BPF-Buffers}
Projekt noch im alpha-Stadium und nicht bereit für das Testen (siehe Abschnitt
\ref{sec:verw_bpf}). Inzwischen ist der Zero-Copy-BPF-Buffers in der neusten
FreeBSD-8.0-Version vorhanden und soll stabil sein. Daher soll die Auswertung 
von Zero-Copy-BPF-Buffers der nächste Schritt sein.
\subsubsection{10Gbit Capturing}
Der Source-Code des \textbf{ixgbe}-Adapter-Treibers~\cite{10gbit_adapter_bsd}
für 10-Gigabit-Netzwerkanschlüsse\footnote{\textbf{ixgbe} - Treiber für Intel
10-Gigabit-Netzwerkadapter} hat eine ähnliche Struktur wie der Source-Code des
\textbf{em}-Treiber. Deshalb sollte die \emph{ringmap}-Software einfach für den
10-Gigabit-Treiber anpassbar sein. Dafür muss aber eine Arbeit vom in
FreeBSD-Kernel-Programming kompetenten Spezialisten geleistet werden, denn
beide Treiber sind nicht vollkommen identisch, und deshalb sollte dem
Softwareentwickler alle Details von BSD-Treiber-Programming bekannt werden, um
die Portierung auf 10Gbit zu machen. Auch hierzu biete ich meine Hilfe an.\\\\
