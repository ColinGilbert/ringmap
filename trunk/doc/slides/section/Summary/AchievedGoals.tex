\begin{frame}
\frametitle{Achieved goals}
\begin{itemize}
	\item \textbf{Enhanced Capturing-Performance}
		\begin{itemize}
			\item very low  system load (below $12\%$) and very low
				packet loss  (below  $0.02\%$)
				\begin{itemize}
					\item during capturing smallest  64-Bytes packets
					\item maximal reached packet-rate ($ > 450000 pkts/sec$)
				\end{itemize}
		\end{itemize}
	\item \textbf{Stability and usability of implemented software}
		\begin{itemize}
			\item during all experiments, no \emph{kernel panics} and no \emph{segmentation faults} occurred
			\item very simple install and remove
				\begin{itemize}
					\item two Shell-Scripts to installing  and removing the   
						\emph{ringmap} Capturing Stacks
				\end{itemize}
			\item Libpcap-applications don't require modification in order to run with the \emph{ringmap}
		\end{itemize}
\end{itemize}
\end{frame}

\begin{frame}
\frametitle{Achieved goals 2}
\begin{itemize}
	\item \textbf{Ringmap is easily portable to the other network controllers}
		\begin{itemize}
			\item The software contains hardware dependent and hardware
				independent parts. Only hardware dependent part require
				modifications.
			\item The generic driver and libpcap should contain a few hooks for calling
				\emph{ringmap}-functions.
		\end{itemize}
	\item \textbf{Packet Filtering}
		\begin{itemize}
			\item Packet filtering can be accomplished using both
				\emph{libpcap}- and kernel-BPF.
		\end{itemize}
\end{itemize}
\end{frame}

\begin{frame}
\frametitle{Achieved goals 3}
\begin{itemize}
	\item \textbf{Multithreaded Capturing}
		\begin{itemize}
			\item Multiple applications can capture from the same interface. 
		\end{itemize}
	\item \textbf{Partly ported to the 10GbE controller}
		\begin{itemize}
			\item Currently only one queue is used while capturing. The work on
				supporting multiqueue is in progress.
		\end{itemize}
\end{itemize}
\end{frame}

