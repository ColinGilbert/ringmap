\begin{frame}
	\begin{center}
	\huge{Achieved goals}
	\end{center}
\end{frame}


\begin{frame}
\frametitle{Achieved goals 1}
\begin{itemize}
	\item \textbf{Portability}
		\begin{itemize}
			\item Ringmap is easily portable to other ethernet controllers.
				Only hardware dependent part of code require modifications.
			\item The generic driver and libpcap contain a few hooks for calling
				\emph{ringmap}-functions.\newline
		\end{itemize}
	\item \textbf{Packet Filtering}
		\begin{itemize}
			\item Packet filtering can be accomplished using both
				\emph{libpcap}- and kernel-BPF.
		\end{itemize}
\end{itemize}
\end{frame}


\begin{frame}
\frametitle{Achieved goals 2}
\begin{itemize}
	\item \textbf{Multithreaded Capturing}
		\begin{itemize}
			\item Multiple applications can capture the packets from the same
				interface.\newline
		\end{itemize}
	\item \textbf{Partly ported to the 10GbE controller}
		\begin{itemize}
			\item Currently only one queue is used while capturing. The work on
				supporting multiqueue is in progress.
		\end{itemize}
\end{itemize}
\end{frame}


\begin{frame}
\frametitle{Achieved goals 3}
\begin{itemize}
	\item \textbf{Enhanced Capturing-Performance}
		\begin{itemize}
			\item very low  system load (below $12\%$) and very low
				packet loss  (below  $0.02\%$)\newline
		\end{itemize}
	\item \textbf{Usability of implemented software}
		\begin{itemize}
			\item Libpcap-applications don't require modification in order to run with the \emph{ringmap}
		\end{itemize}
\end{itemize}
\end{frame}
